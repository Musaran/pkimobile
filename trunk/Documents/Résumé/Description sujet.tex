\documentclass[frenchb, a4paper]{llncs}
\usepackage[T1]{fontenc}
\usepackage[latin1]{inputenc}
\usepackage{times}
\usepackage{url}
\usepackage{version}
\usepackage[francais]{babel}
\usepackage{graphicx}
\usepackage{float}
\usepackage {prettyref}
\usepackage {prettyref}
\usepackage{version}
\DeclareGraphicsExtensions{.jpg,.mps,.pdf,.png,.gif, .ps, .eps}
\urldef{\mailsa}\path|jeanphilippe.blaise@umail.univ-metz.fr|
\urldef{\mailsb}\path|william.jouot@umail.univ-metz.fr|
\newcommand{\keywords}[1]{\par\addvspace\baselineskip
\noindent\keywordname\enspace\ignorespaces#1}
\newcommand{\noun}[1]{\textsc{#1}}
\newtheorem{algorithm}{Algorithm}
\begin{document}
\title{Comparaison fonctionnelle et exp�rimentale d'une PKI sur mobile et de Kerberos sur mobile. Mise en oeuvre et comparaison rigoureuse des r�sultats exp�rimentaux. Automatisation du d�ploiement.}
\author{Jean-Philippe Blaise, Willian Jouot}
\institute{Master SSIC, Metz University,\\
Ile du Saulcy, 57045 Metz, France\\
\mailsa\\
\mailsb\\
}
\maketitle
%\begin {abstract}
%\linespread{1.5cm}
%Ins�rer ici quelques lignes de r�sum�
%\end{abstract}

\begin{description}
	\item[Mots-cl�s :] 
{PKI, Kerberos, mobile, mise en oeuvre, d�ploiement}
\end{description}

\section{Probl�matique}
A l'heure actuelle, les t�l�phones ainsi que les smartphones se d�mocratisent de plus en plus, et il y a donc de plus en plus de donn�es qui transitent via ces appareils, que cela soit via le wifi, bluetooth, ou directement le r�seau t�l�phonique. On commence ainsi � faire des achats directement depuis son t�l�phone, o� encore envoyer des informations personnels, comme des mots de passe, ou des documents que l'on veut garder secret. Malheureusement, toutes ces donn�es transitent plus ou moins en clair dans l'air, � la port�e de n'importe qui. Il faut donc s�curiser les donn�es. Quelles solutions existent ? Des syst�mes comme PKI ont d�j� fait leurs preuves sur nos ordinateurs, mais existe-t-il des solutions concr�tes pour nos mobiles ?
\section{Travaux existants dans la litt�rature scientifique}
Il existe de nombreux travaux dans le domaine, notamment pour la PKI. Le premier est bas� sur une PKI pour un syst�me de paiement\cite{Marko2008}. Un autre document propose une solution d'impl�mentation d'une wireless PKI (WPKI)\cite{YongLee2006}. 
Beaucoup moins d'article parlant de  Kerberos existe � ce jour. Quelques uns sont interessant comme ces articles parlant d'une authentification  Kerberos par billets "r�utilisables"\cite{Anish2010,Yaohui2009}.
\section{Premi�res critiques des travaux existants}

\section{Objectifs et perspectives du projet de synth�se}
L'objectif est d'impl�menter le d�ploiement automatique d'un syst�me de certification pour mobile. Pour cela, il faudra comparer le fonctionnement d'une PKI et de Kerberos aussi bien de mani�re exp�rimentale que fonctionnelle.
Nous nous focaliseront sur une base th�orique pour commenter les documents existants. Et a partir de ce qu'on nous aurons retenu, mettre en oeuvre une PKI pour mobile de la mani�re la plus efficace possible.


\bibliographystyle{ieeetr}
\bibliography{biblio}
\end{document}