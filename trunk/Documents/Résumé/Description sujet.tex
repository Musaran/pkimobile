\documentclass[frenchb, a4paper]{llncs}
\usepackage[T1]{fontenc}
\usepackage[latin1]{inputenc}
\usepackage{times}
\usepackage{url}
\usepackage{version}
\usepackage[francais]{babel}
\usepackage{graphicx}
\usepackage{float}
\usepackage {prettyref}
\usepackage {prettyref}
\usepackage{version}
\DeclareGraphicsExtensions{.jpg,.mps,.pdf,.png,.gif, .ps, .eps}
\urldef{\mailsa}\path|jeanphilippe.blaise@umail.univ-metz.fr|
\urldef{\mailsb}\path|william.jouot@umail.univ-metz.fr|
\newcommand{\keywords}[1]{\par\addvspace\baselineskip
\noindent\keywordname\enspace\ignorespaces#1}
\newcommand{\noun}[1]{\textsc{#1}}
\newtheorem{algorithm}{Algorithm}
\begin{document}
\title{Comparaison fonctionnelle et exp�rimentale d'une PKI sur mobile et de Kerberos sur mobile. Mise en oeuvre et comparaison rigoureuse des r�sultats exp�rimentaux. Automatisation du d�ploiement.}
\author{Jean-Philippe Blaise, Willian Jouot}
\institute{Master SSIC, Metz University,\\
Ile du Saulcy, 57045 Metz, France\\
\mailsa\\
\mailsb\\
}
\maketitle
%\begin {abstract}
%\linespread{1.5cm}
%Ins�rer ici quelques lignes de r�sum�
%\end{abstract}

\begin{description}
	\item[Mots-cl�s :] 
{PKI, Kerberos, mobile}
\end{description}

\section{Probl�matique}
\section{Travaux existants dans la litt�rature scientifique}

\section{Premi�res critiques des travaux existants}

\section{Objectifs et perspectives du projet de synth�se}

\begin{thebibliography}{10}

\bibitem {Meisels02} A. Meisels and Eliezer Kaplansky. Scheduling Agents Distributed Timetabling Problems, in: the 4th International Conference on the Practice and Theory 
of Automated Timetabling, Gent, Belgium, 2002, Pages 166-180. 

\bibitem {Faltings05} B. Faltings and M. Yokoo. Introduction: Special Issue on Distributed Constraint Satisfaction. Artificial Intelligence, Volume 161, Issues 1-2, January 2005, Pages 29-44.

\end{thebibliography}


\end{document}